% Template for PLoS
% Version 3.5 March 2018
%
% % % % % % % % % % % % % % % % % % % % % %
%
% -- IMPORTANT NOTE
%
% This template contains comments intended
% to minimize problems and delays during our production
% process. Please follow the template instructions
% whenever possible.
%
% % % % % % % % % % % % % % % % % % % % % % %
%
% Once your paper is accepted for publication,
% PLEASE REMOVE ALL TRACKED CHANGES in this file
% and leave only the final text of your manuscript.
% PLOS recommends the use of latexdiff to track changes during review, as this will help to maintain a clean tex file.
% Visit https://www.ctan.org/pkg/latexdiff?lang=en for info or contact us at latex@plos.org.
%
%
% There are no restrictions on package use within the LaTeX files except that
% no packages listed in the template may be deleted.
%
% Please do not include colors or graphics in the text.
%
% The manuscript LaTeX source should be contained within a single file (do not use \input, \externaldocument, or similar commands).
%
% % % % % % % % % % % % % % % % % % % % % % %
%
% -- FIGURES AND TABLES
%
% Please include tables/figure captions directly after the paragraph where they are first cited in the text.
%
% DO NOT INCLUDE GRAPHICS IN YOUR MANUSCRIPT
% - Figures should be uploaded separately from your manuscript file.
% - Figures generated using LaTeX should be extracted and removed from the PDF before submission.
% - Figures containing multiple panels/subfigures must be combined into one image file before submission.
% For figure citations, please use "Fig" instead of "Figure".
% See http://journals.plos.org/plosone/s/figures for PLOS figure guidelines.
%
% Tables should be cell-based and may not contain:
% - spacing/line breaks within cells to alter layout or alignment
% - do not nest tabular environments (no tabular environments within tabular environments)
% - no graphics or colored text (cell background color/shading OK)
% See http://journals.plos.org/plosone/s/tables for table guidelines.
%
% For tables that exceed the width of the text column, use the adjustwidth environment as illustrated in the example table in text below.
%
% % % % % % % % % % % % % % % % % % % % % % % %
%
% -- EQUATIONS, MATH SYMBOLS, SUBSCRIPTS, AND SUPERSCRIPTS
%
% IMPORTANT
% Below are a few tips to help format your equations and other special characters according to our specifications. For more tips to help reduce the possibility of formatting errors during conversion, please see our LaTeX guidelines at http://journals.plos.org/plosone/s/latex
%
% For inline equations, please be sure to include all portions of an equation in the math environment.
%
% Do not include text that is not math in the math environment.
%
% Please add line breaks to long display equations when possible in order to fit size of the column.
%
% For inline equations, please do not include punctuation (commas, etc) within the math environment unless this is part of the equation.
%
% When adding superscript or subscripts outside of brackets/braces, please group using {}.
%
% Do not use \cal for caligraphic font.  Instead, use \mathcal{}
%
% % % % % % % % % % % % % % % % % % % % % % % %
%
% Please contact latex@plos.org with any questions.
%
% % % % % % % % % % % % % % % % % % % % % % % %

\documentclass[10pt,letterpaper]{article}
\usepackage[top=0.85in,left=2.75in,footskip=0.75in]{geometry}

% amsmath and amssymb packages, useful for mathematical formulas and symbols
\usepackage{amsmath,amssymb}

% Use adjustwidth environment to exceed column width (see example table in text)
\usepackage{changepage}

% Use Unicode characters when possible
\usepackage[utf8x]{inputenc}

% textcomp package and marvosym package for additional characters
\usepackage{textcomp,marvosym}

% cite package, to clean up citations in the main text. Do not remove.
% \usepackage{cite}

% Use nameref to cite supporting information files (see Supporting Information section for more info)
\usepackage{nameref,hyperref}

% line numbers
\usepackage[right]{lineno}

% ligatures disabled
\usepackage{microtype}
\DisableLigatures[f]{encoding = *, family = * }

% color can be used to apply background shading to table cells only
\usepackage[table]{xcolor}

% array package and thick rules for tables
\usepackage{array}

% create "+" rule type for thick vertical lines
\newcolumntype{+}{!{\vrule width 2pt}}

% create \thickcline for thick horizontal lines of variable length
\newlength\savedwidth
\newcommand\thickcline[1]{%
  \noalign{\global\savedwidth\arrayrulewidth\global\arrayrulewidth 2pt}%
  \cline{#1}%
  \noalign{\vskip\arrayrulewidth}%
  \noalign{\global\arrayrulewidth\savedwidth}%
}

% \thickhline command for thick horizontal lines that span the table
\newcommand\thickhline{\noalign{\global\savedwidth\arrayrulewidth\global\arrayrulewidth 2pt}%
\hline
\noalign{\global\arrayrulewidth\savedwidth}}


% Remove comment for double spacing
%\usepackage{setspace}
%\doublespacing

% Text layout
\raggedright
\setlength{\parindent}{0.5cm}
\textwidth 5.25in
\textheight 8.75in

% Bold the 'Figure #' in the caption and separate it from the title/caption with a period
% Captions will be left justified
\usepackage[aboveskip=1pt,labelfont=bf,labelsep=period,justification=raggedright,singlelinecheck=off]{caption}
\renewcommand{\figurename}{Fig}

% Use the PLoS provided BiBTeX style
% \bibliographystyle{plos2015}

% Remove brackets from numbering in List of References
\makeatletter
\renewcommand{\@biblabel}[1]{\quad#1.}
\makeatother



% Header and Footer with logo
\usepackage{lastpage,fancyhdr,graphicx}
\usepackage{epstopdf}
%\pagestyle{myheadings}
\pagestyle{fancy}
\fancyhf{}
%\setlength{\headheight}{27.023pt}
%\lhead{\includegraphics[width=2.0in]{PLOS-submission.eps}}
\rfoot{\thepage/\pageref{LastPage}}
\renewcommand{\headrulewidth}{0pt}
\renewcommand{\footrule}{\hrule height 2pt \vspace{2mm}}
\fancyheadoffset[L]{2.25in}
\fancyfootoffset[L]{2.25in}
\lfoot{\today}

%% Include all macros below

\newcommand{\lorem}{{\bf LOREM}}
\newcommand{\ipsum}{{\bf IPSUM}}


% tightlist command for lists without linebreak
\providecommand{\tightlist}{%
  \setlength{\itemsep}{0pt}\setlength{\parskip}{0pt}}


% Pandoc citation processing
\newlength{\cslhangindent}
\setlength{\cslhangindent}{1.5em}
\newlength{\csllabelwidth}
\setlength{\csllabelwidth}{3em}
\newlength{\cslentryspacingunit} % times entry-spacing
\setlength{\cslentryspacingunit}{\parskip}
% for Pandoc 2.8 to 2.10.1
\newenvironment{cslreferences}%
  {}%
  {\par}
% For Pandoc 2.11+
\newenvironment{CSLReferences}[2] % #1 hanging-ident, #2 entry spacing
 {% don't indent paragraphs
  \setlength{\parindent}{0pt}
  % turn on hanging indent if param 1 is 1
  \ifodd #1
  \let\oldpar\par
  \def\par{\hangindent=\cslhangindent\oldpar}
  \fi
  % set entry spacing
  \setlength{\parskip}{#2\cslentryspacingunit}
 }%
 {}
\usepackage{calc}
\newcommand{\CSLBlock}[1]{#1\hfill\break}
\newcommand{\CSLLeftMargin}[1]{\parbox[t]{\csllabelwidth}{#1}}
\newcommand{\CSLRightInline}[1]{\parbox[t]{\linewidth - \csllabelwidth}{#1}\break}
\newcommand{\CSLIndent}[1]{\hspace{\cslhangindent}#1}

\usepackage{lineno}
\usepackage[T1]{fontenc}
\linenumbers
\newcommand{\beginsupplement}{\setcounter{table}{0}  \renewcommand{\thetable}{S\arabic{table}}
\newfloat{text}{thp}{}
\floatname{text}{Text}
\renewcommand{\figurename}{}
\newcommand{\newfigurename}{Hello}
\setcounter{figure}{0} \renewcommand{\thefigure}{S\arabic{figure}}}
\usepackage{nameref,hyperref}
\usepackage{caption}
\usepackage{booktabs}
\usepackage{longtable}
\usepackage{array}
\usepackage{multirow}
\usepackage{wrapfig}
\usepackage{float}
\usepackage{colortbl}
\usepackage{pdflscape}
\usepackage{tabu}
\usepackage{threeparttable}
\usepackage{threeparttablex}
\usepackage[normalem]{ulem}
\usepackage{makecell}
\usepackage{xcolor}


\usepackage{forarray}
\usepackage{xstring}
\newcommand{\getIndex}[2]{
  \ForEach{,}{\IfEq{#1}{\thislevelitem}{\number\thislevelcount\ExitForEach}{}}{#2}
}

\setcounter{secnumdepth}{0}

\newcommand{\getAff}[1]{
  \getIndex{#1}{London School of Hygiene \& Tropical
Medicine,CMMID,Karlsruhe Institute of Technology Institute of Economic
Theory and Statistics,Max Planck Institute for Multidisciplinary
Sciences,UK Health Security Agency,Imperial College London}
}

\begin{document}
\vspace*{0.2in}


% Title must be 250 characters or less.
\begin{flushleft}
{\Large
\textbf\newline{Comparing human and model-based forecasts of COVID-19 in
Germany and
Poland} % Please use "sentence case" for title and headings (capitalize only the first word in a title (or heading), the first word in a subtitle (or subheading), and any proper nouns).
}
\newline
% Insert author names, affiliations and corresponding author email (do not include titles, positions, or degrees).
\\
Nikos I. Bosse\textsuperscript{\getAff{London School of Hygiene \&
Tropical Medicine}, \getAff{CMMID}}\textsuperscript{*},
Sam Abbott\textsuperscript{\getAff{London School of Hygiene \& Tropical
Medicine}, \getAff{CMMID}},
Johannes Bracher\textsuperscript{\getAff{Karlsruhe Institute of
Technology Institute of Economic Theory and Statistics}},
Habakuk Hain\textsuperscript{\getAff{Max Planck Institute for
Multidisciplinary Sciences}},
Billy J. Quilty\textsuperscript{\getAff{London School of Hygiene \&
Tropical Medicine}, \getAff{CMMID}},
Mark Jit\textsuperscript{\getAff{London School of Hygiene \& Tropical
Medicine}, \getAff{CMMID}},
Centre for the Mathematical Modelling of Infectious Diseases COVID-19
Working Group\textsuperscript{\getAff{London School of Hygiene \&
Tropical Medicine}, \getAff{CMMID}},
Edwin van Leeuwen\textsuperscript{\getAff{UK Health Security
Agency}, \getAff{London School of Hygiene \& Tropical Medicine}},
Anne Cori\textsuperscript{\getAff{Imperial College London}},
Sebastian Funk\textsuperscript{\getAff{London School of Hygiene \&
Tropical Medicine}, \getAff{CMMID}}\\
\bigskip
\textbf{\getAff{London School of Hygiene \& Tropical
Medicine}}Department of Infectious Disease Epidemiology, London School
of Hygiene \& Tropical Medicine, London, United Kingdom\\
\textbf{\getAff{CMMID}}Centre for the Mathematical Modelling of
Infectious Diseases (members of the CMMID COVID-19 working group are
listed in S1 Acknowledgements), London, United Kingdom\\
\textbf{\getAff{Karlsruhe Institute of Technology Institute of Economic
Theory and Statistics}}Institute of Economic Theory and Statistics,
Karlsruhe Institute of Technology, Karlsruhe, Germany\\
\textbf{\getAff{Max Planck Institute for Multidisciplinary Sciences}}Max
Planck Institute for Multidisciplinary Sciences, Göttingen, Germany\\
\textbf{\getAff{UK Health Security Agency}}UK Health Security Agency,
London United Kingdom\\
\textbf{\getAff{Imperial College London}}MRC Centre for Outbreak
Analysis and Modelling, Department of Infectious Disease Epidemiology,
School of Public Health, Imperial College London, London, United
Kingdom\\
\bigskip
* nikos.bosse@lshtm.ac.uk\\
\end{flushleft}
% Please keep the abstract below 300 words
\section*{Abstract}
Forecasts based on epidemiological modelling have played an important
role in shaping public policy throughout the COVID-19 pandemic. This
modelling combines knowledge about infectious disease dynamics with the
subjective opinion of the researcher who develops and refines the model
and often also adjusts model outputs. Developing a forecast model is
difficult, resource- and time-consuming. It is therefore worth asking
what modelling is able to add beyond the subjective opinion of the
researcher alone. To investigate this, we analysed different real-time
forecasts of cases of and deaths from COVID-19 in Germany and Poland
over a 1-4 week horizon submitted to the German and Polish Forecast Hub.
We compared crowd forecasts elicited from researchers and volunteers,
against a) forecasts from two semi-mechanistic models based on common
epidemiological assumptions and b) the ensemble of all other models
submitted to the Forecast Hub. We found crowd forecasts, despite being
overconfident, to outperform all other methods across all forecast
horizons when forecasting cases (weighted interval score relative to the
Hub ensemble 2 weeks ahead: 0.89). Forecasts based on computational
models performed comparably better when predicting deaths (rel. WIS
1.26), suggesting that epidemiological modelling and human judgement can
complement each other in important ways.

% Please keep the Author Summary between 150 and 200 words
% Use first person. PLOS ONE authors please skip this step.
% Author Summary not valid for PLOS ONE submissions.
\section*{Author summary}
Mathematical models of COVID-19 have played a key role in informing
governments across the world. While mathematical models are informed by
our knowledge of infectious disease dynamics, they are ultimately
developed and iteratively adjusted by the researchers and shaped by
their subjective opinions. To investigate what modelling is able to add
beyond the subjective opinion of the researcher alone, we compared human
forecasts with model-based predictions of COVID-19 cases and deaths
submitted to the so-called German/Polish Forecast Hub (which collates a
variety of models from a range of teams). \textbar{} We found that our
human forecasts consistently outperformed an aggregate of all available
model-based forecasts when predicting cases, but not when predicting
deaths. Our findings suggest that human insight may be most valuable
when forecasting highly uncertain quantities, which depend on many
factors that are hard to model using equations, while mathematical
models may be most useful in settings like predicting deaths, where
leading indicators with a clear connection to the target variable are
available. This potentially has very relevant policy implications, as
agencies informing policy-makers could benefit from routinely eliciting
human forecasts in addition to model-based predictions to inform
policies.

\linenumbers

% Use "Eq" instead of "Equation" for equation citations.
\hypertarget{introduction}{%
\section{Introduction}\label{introduction}}

\cite{mcgowanCollaborativeEffortsForecast2019}

Infectious disease modelling has a long tradition and has helped inform
public health decisions both through scenario modelling, as well as
actual forecasts of (among others) influenza {[}e.g. 1,2--4{]}, dengue
fever {[}e.g. 5,6,7{]}, ebola {[}e.g. 8,9{]}, chikungunya {[}e.g.
10,11{]} and now COVID-19 {[}e.g. 12,13--17{]}. Applications of
epidemiological models differ in the way they make statements about the
future. Forecasts aim to predict the future as it will occur, while
scenario modelling and projections aim to represent what the future
could look like under certain scenario assumptions or if conditions
stayed the same as they were in the past. Forecasts can be judged by
comparing them against observed data. Since it is much harder to fairly
assess the accuracy and usefulness of projections and scenario modelling
in the same way, this work focuses on forecasts, which represent only a
subset of all epidemiological modelling.

Since March 2020, forecasts of COVID-19 from multiple teams have been
collected, aggregated and compared by Forecast Hubs such as the US
Forecast Hub {[}13,14{]}, the German and Polish Forecast Hub {[}15,16{]}
and the European Forecast Hub {[}17{]}. Often, different individual
forecasts are combined into a single forecast, e.g.~by taking the mean
or median of all forecasts. These ensemble forecasts usually tend to
perform better and more consistently than individual forecasts (see e.g.
{[}6{]}; {[}18{]}).

Individual computational models usually rely to varying degrees on
mechanistic assumptions about infectious disease dynamics (such as
SIR-type compartmental models that aim to represent how individuals move
from being susceptible to infected and then recovered or dead). Some are
more statistical in nature (such as time series models that detect
statistical patterns without explicitly modelling disease dynamics). How
exactly such a mathematical or computational model is constructed and
which assumptions are made depends on subjective opinion and judgement
of the researcher who develops and refines the model. Models are
commonly adjusted and improved based on whether the model output looks
plausible to the researchers involved.

The process of model construction and refinement is laborious and
time-consuming, and it is therefore worth asking what modelling can add
beyond the subjective judgment of the researcher alone. In this work, we
ask this question specifically in the context of predictive performance,
and set aside other advantages of epidemiological modelling (such as
reproducibility or the ability to obtain a deeper fundamental
understanding of how diseases spread). One natural way to do this is to
compare the predictive performance of forecasts based on computational
models (``model-based forecasts'') against forecasts made by individual
humans without explicit use of a computer model (``direct human
forecasts'') or a combination of multiple such forecasts (``crowd
forecasts'').

Previous work has examined such direct human forecasts in various
contexts, such as geopolitics {[}19,20{]}, meta-science {[}21,22{]},
sports {[}23{]} and epidemiology {[}11,24,25{]}. Several prediction
platforms {[}26--28{]} and prediction markets {[}29{]} have been created
to collate expert and non-expert predictions. However, with the notable
exception of {[}11{]}, these forecasts were not designed to be evaluated
alongside model-based forecasts and usually follow their own (often
binary) prediction formats. Direct human forecasts may be able to take
into account insights and relationships between variables which are hard
to specify using epidemiological models. However, it is not entirely
clear in which situations human forecasts perform well or badly. For
example, {[}11{]} found that humans could outperform computer models at
predicting the 2014/15 and 2015/16 flu season in the US, a setting where
the disease was well known and information about previous seasons was
available. However, humans tended to do slightly worse at predicting the
2014/15 outbreak of chikungunya in the Americas, a disease previously
largely unobserved and unknown in these regions at the time.

In this study, we analyse the performance of direct human forecasts
relative to model-based forecasts and discuss the added benefit of
epidemiological modelling over human judgement alone. As a case study,
we use different forecasts, involving varying degrees of human
intervention, which we submitted in real time to the German and Polish
Forecast Hub. In contrast to {[}11{]} we elicited not only point
predictions, but full predictive distributions (``probabilistic
forecasts'', see e.g. {[}30{]}) from participants. This allows us to
compare not only predictive accuracy, but also how well human
forecasters and model-based forecasts were able to quantify forecast
uncertainty.

\hypertarget{methods}{%
\section{Methods}\label{methods}}

\hypertarget{ethics-statement}{%
\subsection{Ethics statement}\label{ethics-statement}}

This study has been approved by the London School of Hygiene \& Tropical
Medicine Research Ethics Committee (reference number 22290). Consent
from participants was obtained in written form.

\hypertarget{overview}{%
\subsection{Overview}\label{overview}}

We created and submitted the following forecasts to the German and
Polish Forecast Hub: 1) a direct human forecast (henceforth called
``crowd forecast''), elicited from participants through a web
application {[}31{]} and 2) two semi-mechanistic model-based forecasts
(``renewal model'' and ``convolution model'') informed by basic
assumptions about COVID-19 epidemiology. While the two semi-mechanistic
forecasts were necessarily shaped by our implicit assumptions and
decisions, they were designed such as to minimise the amount of human
intervention involved. For example, we refrained from adjusting model
outputs or refining the models based on past performance. Forecasts were
created in real time over a period of 21 weeks from October 12th 2020
until March 1st 2021 and submitted to the German and Polish Forecast hub
{[}15,16{]}. All code and tools necessary to generate the forecasts and
make a forecast submission are available in the
\texttt{covid.german.forecasts} R package {[}32{]}. This repository also
contains a record of all forecasts submitted to the German and Polish
Forecast Hub. Forecasts were evaluated using a variety of scoring
metrics and compared among each other and against an ensemble of all
other models submitted to the German and Polish Forecast Hub.

\hypertarget{forecast-targets-and-interaction-with-the-german-and-polish-forecast-hub}{%
\subsection{Forecast targets and interaction with the German and Polish
Forecast
Hub}\label{forecast-targets-and-interaction-with-the-german-and-polish-forecast-hub}}

The German and Polish Forecast Hub (now mostly merged into the {[}17{]})
elicits predictions for various COVID-19 related forecast targets from
different research groups every week. Forecasts had to be made every
Monday (with submissions allowed until Tuesday 3pm) and were permitted
to use any data that was available by Monday 11.59pm. We submitted
forecasts for incident and cumulative weekly reported numbers of cases
of and deaths from COVID-19 on a national level in Germany and Poland
over a one to four week forecast horizon. Forecasts were submitted on
Mondays, but weeks were defined as ending on a Saturday (and starting on
Sunday), meaning that forecast horizons were in fact 5, 12, 19 and 26
days. Submissions were required in a quantile-based format with 23
quantiles of each output measure at levels 0.01, 0.025, 0.05, 0.10,
0.15, \ldots, 0.95, 0.975, 0.99. Forecasts submitted to the Forecast Hub
were combined into different ensembles every week, with the median
ensemble (i.e., the \(\alpha\)-quantile of the ensemble is given by the
median of all submitted \(\alpha\)-quantiles) being the default ensemble
shown on all official Forecast hub visualisations
(https://kitmetricslab.github.io/forecasthub/forecast).

Data on daily reported test positive cases and deaths linked to COVID-19
were provided by the organisers of the German and Polish Forecast hub.
Until December 14th, 2020, these data were sourced from the European
Centre for Disease Control {[}33{]}. After ECDC stopped publishing daily
data, observations were sourced from the Robert Koch Institute (RKI) and
the Polish Ministry of Health for the remainder of the submission period
{[}34{]}. These data are subject to reporting artefacts, (such as for
example delayed case reporting in Poland on the 24th November,
{[}35{]}), changes in reporting over time, and variation in testing
regimes (for example in Germany from the 11th of November on, {[}36{]}).
The ECDC data as well as the data published by the Polish Ministry of
Health were also subject to data revisions, although most of them (with
a notable exception of a data update for October 12 2020 in Germany)
only affected daily, not weekly data (see
\nameref{fig:daily-truth-update} and \nameref{fig:weekly-truth-update}
Figs).

\hypertarget{crowd-forecasts}{%
\subsection{Crowd forecasts}\label{crowd-forecasts}}

Our crowd forecasts were created as an ensemble of forecasts made by
individual participants every week through a web application
(https://cmmid-lshtm.shinyapps.io/crowd-forecast/). Weekly forecasts had
to be submitted before Tuesday 12pm every week, but participants were
asked to only use any information or data that was already available by
Monday night. The application was built using the \texttt{shiny} and
\texttt{golem} R packages {[}37,38{]} and is available in the
\texttt{crowdforecastr} R package {[}31{]}. To make a forecast in the
application participants could select a predictive distribution (with
the default being log-normal) to represent the probability that the
forecasted quantity took certain values. Median and width of the
uncertainty could be adjusted by either interacting with a figure
showing their forecast or providing numerical values (see screenshot in
\nameref{fig:screenshot}). The default shown was a repetition of the
last known observation with constant uncertainty around it computed as
the standard deviation of the last four changes in weekly log observed
forecasts (i.e.~as
\(\sigma(log(value4) - log(value3), log(value3) - log(value2), \ldots )\)).
A comparison of the crowd forecasts against the default baseline shown
in the application is displayed in \nameref{fig:compare-forecasters}.
Our interface also allowed participants to view past observations based
on the hub data, as well as their forecasts, on a logarithmic scale and
presented additional contextual COVID-19 data sourced from {[}39{]}.
These data included, for example, notifications of both test positive
COVID-19 cases and COVID-19 linked deaths and the number of COVID-19
tests conducted over time. From November 26 2020 on we displayed weekly
small reports with a visualisation of past forecasts and scores on our
website, epiforecasts.io.

Forecasts were stored in a Google Sheet and downloaded, cleaned and
processed every week for submission to the Forecast Hub. If a forecaster
had submitted multiple predictions for a single target, only the latest
submission was kept. Information on the chosen distribution as well as
the parameters for median and width were used to obtain the required set
of 23 quantiles from that distribution. Forecasts from all forecasters
were then aggregated using an unweighted quantile-wise mean (i.e., the
\(\alpha\)-quantile of the ensemble is given by the mean of all
submitted \(\alpha\)-quantiles). To avoid issues with users trying out
the app and submitting a random forecast, we required that a forecaster
needed to make a forecast for at least two targets for a given forecast
in order to be included in the crowd forecast ensemble. On a few
occasions we deleted forecasts that were clearly the result of a user or
software error (such as for example forecasts that were zero
everywhere).

Participants were recruited mostly within the Centre of Mathematical
Modeling of Infectious Diseases at the London School of Hygiene \&
Tropical Medicine, but participants were also invited personally or via
social media to submit predictions. Depending on whether they had a
background in either statistics, forecasting or epidemiology,
participants were asked to self-identify as `experts' or `non-experts'.

\hypertarget{model-based-forecasts}{%
\subsection{Model-based forecasts}\label{model-based-forecasts}}

We used two Bayesian semi-mechanistic models from the \texttt{EpiNow2} R
package (version 1.3.3) as our model-based forecasts {[}40{]}. The first
of these models, here called ``renewal model'', used the renewal
equation {[}41{]} to predict reported cases and deaths (see details in
\nameref{txt:details-models}). It estimated the effective reproduction
number \(R_t\) (the average number of people each person infected at
time t is expected to infect in turn) and modelled future infections as
a weighted sum of past infection multiplied by \(R_t\). \(R_t\) was
assumed to stay constant beyond the forecast date, roughly corresponding
to continuing the latest exponential trend in infections. On the 9th of
November we altered the date when \(R_t\) was assumed to be constant
from two weeks prior to the date of the forecast to the forecast date,
which we found to yield a more stable \(R_t\) estimate. Reported case
and death notifications were obtained by convolving predicted infections
over data-based delay distributions {[}40,42--44{]} to model the time
between infection and report date. The renewal model was used to predict
cases as well as deaths with forecasts being generated for each target
separately. Death forecasts from the renewal model were therefore not
informed by past cases. One submission of the renewal model on December
28th 2020 was delayed and therefore not included in the official
Forecast hub ensemble.

The second model (``convolution model'', see details in
\nameref{txt:details-models}) was only used to forecast deaths and was
added later, starting December 7th 2020 (with the first forecast from
December 7th suffering from a software bug and therefore disregarded in
all further analyses). The convolution model was submitted, but never
included in the official Forecast hub ensemble due to concerns that it
could be too similar to the renewal model. The convolution model
predicted deaths as a fraction of infected people who would die with
some delay, by using a convolution of reported cases with a distribution
that described the delay from case report to death and a scaling factor
(the case-fatality ratio). Both the renewal and the convolution model
used daily observations and assumed a negative binomial observation
model with a multiplicative day-of-the-week effect {[}40{]}.

Line list data used to inform the prior for the delay from symptom onset
to test positive case report or death in the model-based forecasts was
sourced from {[}45{]} with data available up to the 1st of August. All
model fitting was done using Markov-chain Monte Carlo (MCMC) in stan
{[}46{]} with each location and forecast target being fitted separately.

\hypertarget{analysis}{%
\subsection{Analysis}\label{analysis}}

For the main analysis we focused mostly on two week ahead forecasts, as
COVID-19 forecasts, especially for cases, were in the past found to have
poor predictive performance beyond this horizon {[}15{]}. Forecasts for
cases were scored using the full period from October 2020 until March
2021. To ensure comparability between models, all death forecasts were
scored using only the period from December 14th on, where all models
including the convolution model were available. To ensure robustness of
our results we conducted a sensitivity analysis where all forecasts
(including cases) were scored only over the later period for which all
forecasts were available (see \nameref{fig:agg-performance-all-late} and
\nameref{tab:score-table-late-2} and \nameref{tab:score-table-late-4}s).
Results remained broadly unchanged.

Forecasts were analysed using the following scoring metrics: The
weighted interval score (WIS) {[}47{]}, the absolute error, relative
bias, and empirical coverage of the 50\% and 90\% prediction intervals.
The WIS is a proper scoring rule {[}48{]}, meaning that in expectation
the score is optimised by reporting a predictive distribution that is
identical to the true data-generating distribution. Forecasters are
therefore incentivised to report their true belief about the future. The
WIS can be understood as a generalisation of the absolute error to
quantile-based forecasts (also meaning that smaller values are better)
and can be decomposed into three separate penalties: forecast spread
(i.e.~uncertainty of forecasts), over-prediction and under-prediction.
While the over- and under-prediction components of the WIS capture the
amount of over-prediction and under-prediction in absolute terms, we
also look at a relative tendency to make biased forecasts. The bias
metric {[}9{]} we use captures how much probability mass of the forecast
was above or below the true value (mapped to values between -1 and 1)
and therefore represents a general tendency to over- or under-predict in
relative terms. A value of -1 implies that all quantiles of the
predictive distribution are below the observed value and a value of 1
that all quantiles are above the observed value. Empirical coverage is
the percentage of observed values that fall inside a given prediction
interval (e.g.~how many observed values fall inside all 50\% prediction
intervals). Scoring metrics are explained in more detail in
\nameref{tab:scoring-metrics}. All scores were calculated using the
\texttt{scoringutils} R package {[}49{]}.

At all stages of the evaluation our forecasts were compared to the
median ensemble of all \emph{other} models submitted to the German and
Polish Forecast Hub (``Hub ensemble''). This ``Hub ensemble'' was
retrospectively computed and excludes all our models, leaving on average
five ensemble member models (see \nameref{tab:table-ensemble-versions}
and \nameref{fig:num-ensemble-members}). What we call ``Hub ensemble''
in this article therefore differs from the ``official Hub ensemble''
(here called ``hub-ensemble-realised'') which included crowd forecasts
as well as renewal model forecasts. To enhance interpretability of
scores we mainly report WIS relative to the Hub ensemble in the main
text, i.e.~we divided the average scores for a given model by the
average score achieved by the Hub ensemble on the same set of forecasts
(with values \textgreater1 implying worse and values \textless1 implying
better performance than the Hub ensemble). In addition to comparing our
forecasts against the hub ensemble excluding our models, we also
assessed the impact of our forecasts on the performance of the
forecasting hub by recalculating separate versions of the Hub ensemble
with only some (or all) of our forecasts included. Versions that
included either all of our models (``hub-ensemble-with-all'') or only
one of them (``hub-ensemble-with-X'') were computed retrospectively.

\hypertarget{results}{%
\section{Results}\label{results}}

\hypertarget{crowd-forecast-participation}{%
\subsection{Crowd forecast
participation}\label{crowd-forecast-participation}}

A total number of 32 participants submitted forecasts, 17 of those
self-identified as `expert' in either forecasting or epidemiology. The
median number of forecasters for any given forecast target was 6, the
minimum 2 and the maximum 10. The mean number of submissions from an
individual forecaster was 4.7 but the median number was only one - most
participants dropped out after their first submission. Only two
participants submitted a forecast every single week, both of whom are
authors on this study.

\begin{figure}[H]
\caption{\bf{Visualisation of aggregate performance metrics for forecasts one to four weeks into the future.}}
A, B: mean weighted interval score (WIS, lower indicates better performance) across horizons. WIS is decomposed into its components dispersion, over-prediction and under-prediction. C: Empirical coverage of the 50\% prediction intervals (50\% coverage is perfect). D: Empirical coverage of the 90\% prediction intervals. E: Dispersion (same as in panel A, B). Higher values mean greater dispersion of the forecast and imply ceteris paribus a worse score. F: Bias, i.e. general (relative) tendency to over- or underpredict. Values are between -1 (complete under-prediction) and 1 (complete over-prediction) and 0 ideally. G: Absolute error of the median forecast (lower is better). H. Standard deviation of all WIS values for different horizons
\label{fig:agg-performance-all}
\end{figure}

\begin{figure}[H]
\caption{\bf{Distribution of scores. }}
A: Distribution of weighted interval scores for two week ahead forecasts of the different models and forecast targets. Points denote single forecasts scores, while the shaded area shows an estimated probability density. B: Distribution of WIS separate by country. Black squares indicate median and black circles mean scores.
\label{fig:distribution-scores}
\end{figure}

\begin{figure}[H]
\caption{\bf{Two week ahead forecasts and corresponding scores. }}
A, C: Visualisation of 50\% prediction intervals of two week ahead forecasts against the reported values. Forecasts that were not scored (because there was no complete set of death forecasts available) are greyed out. B, D: Visualisation of corresponding WIS.
\label{fig:forecasts-and-truth}
\end{figure}

\hypertarget{case-forecasts}{%
\subsection{Case Forecasts}\label{case-forecasts}}

For cases, crowd forecasts had a lower mean weighted interval score
(WIS, lower values indicate better performance) than both the renewal
model and the Hub ensemble across all forecast horizons (Fig
\ref{fig:agg-performance-all}A) and locations
(\nameref{fig:performance-locations-rel}A Fig). For two week ahead
forecasts, mean WIS relative to the Hub ensemble (= 1) was 0.89 for
crowd forecasts and 1.40 for the renewal model
(\nameref{tab:score-table-2} Table). Across all forecasting approaches,
locations and forecast horizons, the distribution of WIS values was very
right-skewed, and average performance was heavily influenced by outliers
(see Fig \ref{fig:distribution-scores}). Overall, low variance in
forecast performance was closely linked with good mean performance (Fig
\ref{fig:agg-performance-all}H and \ref{fig:agg-performance-all}A),
suggesting that the ability to avoid large errors was an important
factor in determining overall performance. The impact of outlier values
was especially pronounced for the renewal model, which had more
outliers, as well as the highest standard deviation of WIS values
(standard deviation of the WIS relative to the WIS sd of the Hub
ensemble was 1.54 at the two weeks ahead horizon), while the ensemble of
crowd forecasts (rel. WIS sd 0.76) and the Hub ensemble (= 1) showed
more stable performance.

To varying degrees, all forecasts exhibited trend-following behaviour
and were rarely able to predict a change in trend before it had
happened. For example, all forecasts failed to predict the change in
trend from increase to decrease that happened in November in Germany and
severely overshot reported cases (Fig \ref{fig:forecasts-and-truth}A).
This was most striking for the renewal model, which extrapolated
unconstrained exponential growth based on the recent past of
observations. The Hub ensemble and the crowd forecast, which had both
been under-predicting throughout October, also failed to predict the
change in trend after cases peaked, but less severely so. Human
forecasters, possibly aware of the semi-lockdown announced on November
2nd 2020 {[}50{]} and the change in the testing regime (with stricter
test criteria) on November 11th 2020 {[}36{]}, were fastest to adapt to
the new trend, and the Hub ensemble slowest. In December, cases rose
again in Germany, with all models under-predicting this growth to
varying extents. As in October, the renewal model captured the phase of
exponential growth in cases slightly better than other approaches, but
again overshot when reported case numbers fell over Christmas. The large
variance in predictions in January in Germany (severe under-prediction
followed by severe over-prediction) may in part be caused by the fact
that the renewal model operated on daily data and therefore was
susceptible to fluctuations in daily reporting around Christmas that
would not have influenced on weekly reporting. Similar trends in
performance were evident in Poland, with the crowd forecast quickest at
adapting to the change in trend in November. In general, there were
fewer large outlier forecasts in Poland and in particular the renewal
model performed more in line with other forecasts there.

All forecasting approaches, including the Hub ensemble, were
overconfident, i.e.~they showed lower than nominal coverage (meaning
that 50\% (90\%) prediction intervals generally covered less than 50\%
(90\%) of the actually observed values) (Fig
\ref{fig:agg-performance-all}C and \ref{fig:agg-performance-all}D).
Coverage for all forecasts deteriorated with increasing forecast
horizon, indicating that all forecasting approaches struggled to
quantify uncertainty appropriately for case forecasts. This was
especially an issue for crowd forecasts, which had markedly shorter
prediction intervals (i.e., narrower and more confident predictive
distributions) than other approaches (Fig
\ref{fig:agg-performance-all}E) and only showed a small increase in
uncertainty across forecast horizons. The crowd forecasts prediction
intervals were also noticeably narrower than the default baseline shown
to forecasters in the application (see
\nameref{fig:compare-forecasters}).

In spite of good performance in terms of the absolute error (Fig
\ref{fig:agg-performance-all}G), the narrow forecast intervals led to
forecasts which were severely overconfident (covering only 36\% and 55\%
of all observations with the 50\% and 90\% prediction intervals of all
forecasts made at a two week forecast horizon, and only 5\% and 38\%
four weeks ahead) (Fig \ref{fig:agg-performance-all}C and
\ref{fig:agg-performance-all}D as well as \nameref{tab:score-table-2}
and \nameref{tab:score-table-4}s). Despite worse performance in terms of
absolute error (Fig \ref{fig:agg-performance-all}G), the renewal model
achieved better calibration (comparable to the Hub ensemble), as
uncertainty increased rapidly across forecast horizons. The crowd
forecasts, on the other hand, showed a smaller bias than the renewal
model, but were overconfident.

The renewal model exhibited a noticeable tendency towards
over-predicting reported cases across all horizons. The crowd forecast
tended to over-predict at longer forecast horizons, whereas the Hub
ensemble showed no systematic bias (Fig \ref{fig:agg-performance-all}F).
Regardless of a general relative tendency to over-predict, all
forecasting approaches incurred larger absolute penalties from over-
than from under-prediction (see decomposition of the WIS into absolute
penalties for over-prediction, under-prediction and dispersion in Fig
\ref{fig:agg-performance-all}A and \ref{fig:agg-performance-all}B, as
well as \nameref{tab:score-table-2} and \nameref{tab:score-table-4}s).

Generally, trends in overall performance were broadly similar across
locations (\nameref{fig:performance-locations} and
\nameref{fig:performance-locations-rel} Figs). Due to the differing
population sizes and numbers of notifications in Germany and Poland
absolute scores were difficult to compare directly. However, relative to
the Hub ensemble, the crowd forecasts performed noticeably better in
Germany than in Poland and the renewal model better in Poland than in
Germany (\nameref{fig:performance-locations-rel}A,
\nameref{fig:performance-locations-rel}G,
\nameref{fig:agg-performance-all-Germany} and
\nameref{fig:agg-performance-all-Poland} Figs).

\hypertarget{death-forecasts}{%
\subsection{Death Forecasts}\label{death-forecasts}}

For deaths, the Hub ensemble outperformed the crowd forecasts as well as
our model-based approaches across all forecast horizons and locations
(Figs \ref{fig:agg-performance-all}B and
\nameref{fig:performance-locations}B). Relative WIS values for the
models two weeks ahead were 1.22 (convolution model), 1.26 (crowd
forecast), 1 (Hub ensemble) and 1.79 (renewal model). The crowd
forecasts performed better than the renewal model across all forecast
horizons and locations (Figs \ref{fig:agg-performance-all}B and
\nameref{fig:performance-locations}B), and also better than the
convolution model three and four weeks ahead. Poor performance of the
renewal model, especially at longer horizons, indicates that an approach
that does not know about past cases, but instead estimates and projects
a separate \(R_t\) trace from deaths, does not use the available
information efficiently. The convolution model was able to outperform
both the renewal model and the crowd forecasts at shorter forecast
horizons (where the delay between cases and deaths means that future
deaths are largely informed by present cases), but saw performance
deteriorate at three and four weeks ahead (where case predictions from
the renewal model were increasingly used to inform death predictions)
(Fig \ref{fig:agg-performance-all}B, \nameref{tab:score-table-4}).

As past cases and hospitalisations can be used as predictors, predicting
a change in trend may be easier for deaths than for cases. Even though
all forecasts generally struggled with this, there were some instances
where changing trends were well captured or even anticipated. In Poland,
for example, the Hub ensemble was able to capture or even anticipate the
peak in deaths in December quite well (whereas the renewal model and
crowd forecast did not). The renewal model, which mostly exhibited
trend-following behaviour, correctly predicted another increase in
weekly deaths in mid-January (potentially based on changes in daily
deaths, as the renewal model did not know about past cases). In Germany
in early January, all models predicted a decrease in deaths two to three
weeks before it actually happened. Predictions from the renewal model at
that time were likely strongly influenced by an unexpected drop in
reported deaths in December. The other forecasting approaches and in
particular, the convolution model may have been affected by potentially
under-reported case numbers around Christmas. When the decrease that all
models had predicted to happen in early January failed to materialise,
the renewal model and the crowd forecast noticeably over-corrected and
over-predicted deaths in the following weeks, while the Hub ensemble,
and to a slightly lesser degree, the convolution model were able to
capture the downturn well when it finally happened at the end of
January.

Death forecasts, generally, showed greater coverage of the 50\% and 90\%
prediction intervals than case forecasts and no decrease in coverage
across forecast horizons, indicating that it might be easier to
appropriately quantify uncertainty for death forecasts. The Hub ensemble
had the greatest coverage, with empirical coverage of the 50\% and 90\%
prediction intervals exceeding 50\%, and 90\%, respectively, across all
forecast horizons. Coverage for the crowd forecasts and our model-based
approaches was generally lower than that of the Hub ensemble and mostly
slightly lower than nominal coverage (Fig \ref{fig:agg-performance-all}C
and \ref{fig:agg-performance-all}D). As for cases, the crowd forecast
tended to have the narrowest prediction intervals and uncertainty
increased most slowly across forecast horizons, and the renewal model
forecasts generally were widest. The convolution model had relatively
narrow prediction intervals for short forecast horizons, but had rapidly
(and non-linearly) increasing uncertainty for longer forecast horizons,
driven by increasing uncertainty in the underlying case forecasts.

For deaths, the ensemble of crowd forecasts had a consistent tendency to
over-predict (see Fig \ref{fig:agg-performance-all}F). The convolution
model had a strong tendency to under-predict, with the magnitude of
under-prediction steadily decreasing for longer forecast horizons. The
renewal model (which over-predicted for cases) and the Hub ensemble
slightly tended towards under-prediction. For deaths, absolute over- and
under-prediction penalties were more in line with a general relative
tendency to over- or under-predict than for cases (Fig
\ref{fig:agg-performance-all}A and \ref{fig:agg-performance-all}B, as
well as \nameref{tab:score-table-2} and \nameref{tab:score-table-4}s).

\begin{figure}[H]
\caption{\bf{Relative aggregate performance metrics across forecast horizons for different versions of the Hub median ensemble.}}
“Hub-ensemble” \textit{excludes} all our models, Hub-ensemble-all \textit{includes} all of our models, “Hub-ensemble-realised” is the actual hub-ensemble observed in reality, which includes the renewal model and the crowd forecasts, but not the convolution model. A, B: mean weighted interval score (WIS) across horizons relative to the Hub ensemble (lower values indicate better performance). C, D: Empirical coverage of the 50\% and 90\% prediction intervals minus empirical coverage observed for the Hub ensemble. E: Dispersion relative to the dispersion of the Hub ensemble. Higher values mean greater dispersion of the forecast and imply ceteris paribus a worse score. F: Bias, i.e. general (relative) tendency to over- orunderpredict. Values are between -1 (complete under-prediction) and 1 (complete over-prediction) and 0 ideally. G: Absolute error of the median forecast relative to the Hub ensemble. H. Standard deviation of all WIS values for different horizons relative to the Hub ensemble.
\label{fig:agg-performance-ensemble-rel}
\end{figure}

\hypertarget{contribution-to-the-forecast-hub}{%
\subsection{Contribution to the Forecast
Hub}\label{contribution-to-the-forecast-hub}}

\label{contributions-hub}

Of our three models, only the renewal model and the crowd forecast were
included in the official Forecast Hub median ensemble
(``hub-ensemble-realised''), while the convolution model was never
included as it was deemed too similar to the existing renewal model. In
the official Hub ensemble, there were on average 7.1 models included
(including our own), with a median of 7, a minimum of 4 (on 28 December
2020 over the Christmas period) and a maximum of 10. Versions that
included either all of our models (``hub-ensemble-with-all'') or only
one of them (``hub-ensemble-with-X'') were computed retrospectively. An
overview of all models and ensemble versions is shown in
\nameref{tab:table-ensemble-versions}.

For cases, our contributions (compared to the Hub ensemble without our
contributions) consistently improved performance across all forecasting
horizons (rel. WIS 0.9 two weeks ahead, see
\nameref{tab:score-table-ensemble-2} Table). Contributions from the
crowd forecasts alone also improved performance of the Hub ensemble
across all forecast horizons, while contributions from the renewal model
had a negative effect for longer horizons (rel. WIS 1.02 three weeks
ahead, 1.06 four weeks ahead). The realised ensemble including both
models performed better or equal compared to all versions with only one
model included for up to three weeks ahead, suggesting synergistic
effects. Only for predictions four weeks ahead would removing the
renewal model have improved performance
(\nameref{tab:score-table-ensemble-4}). The realised ensemble performed
comparably to the crowd forecasts for predictions one to two weeks
ahead, and worse for greater forecast horizons.

For deaths, contributions from the renewal model and crowd forecast
together improved performance only for one week ahead predictions and
showed an increasingly negative impact on performance for longer
horizons (rel. WIS of the Hub-ensemble-realised 1.01 two weeks ahead,
1.05 four weeks ahead, \nameref{tab:score-table-ensemble-2} and
\nameref{tab:score-table-ensemble-4}s). Individual contributions from
both the renewal model and the crowd forecast were largely negative,
while a version of the Hub ensemble with only the convolution model
included would have performed consistently better across all forecast
horizons (with the positive impact increasing for longer horizons). This
is especially interesting as the convolution model performed
consistently worse than the pre-existing Hub ensemble (Fig
\ref{fig:agg-performance-all}) and especially worse for longer horizons.

We also considered the impact of our contributions on a version of the
Hub ensemble constructed by taking the quantile-wise mean, rather than
the median. General trends were similar, with the notable exception of
the convolution model, which had a consistently positive impact on the
median ensemble, but a mixed and mostly slightly negative impact on the
mean ensemble (Figs \ref{fig:agg-performance-ensemble-rel}B and
\nameref{fig:agg-performance-ensemble-mean}B). This may happen if a
model is more correct directionally relative to the pre-existing
ensemble, but overshoots in absolute terms, thereby moving the ensemble
too far. For both the mean and the median ensemble, changes in
performance from adding or removing models were of a similar order of
magnitude, suggesting that at least in this instance, with a relatively
small ensemble size, the median ensemble was not necessarily more
`robust' to changes than the mean ensemble. However, the ensemble
version with all our forecasts included (``hub-ensemble-with-all'')
tended to perform relatively better for the median ensemble than the
mean ensemble, suggesting that adding more models may be more beneficial
or `safer' for the median than for the mean ensemble as directional
errors can more easily cancel out than errors in absolute terms.

\hypertarget{discussion}{%
\section{Discussion}\label{discussion}}

Epidemiological forecasting modelling combines knowledge about
infectious disease dynamics with the subjective opinion of the
researcher who develops and refines the model. In this study, we
compared forecasts of cases of and deaths from COVID-19 in Germany and
Poland based purely on human judgement and elicited from a crowd of
researchers and volunteers against forecasts from two semi-mechanistic
epidemiological models. In spite of the small number of participants and
a general tendency to be overconfident, crowd forecasts consistently
outperformed our epidemiological models as well as the Hub ensemble when
forecasting cases but not when forecasting deaths. This suggests that
humans might be relatively good at foreseeing trends that are hard to
model but may struggle to form an intuition for the exact relationship
between cases and deaths.

Past studies have evaluated the performance of model-based forecasting
approaches as well as human experts and non-experts in various contexts.
However, most of these studies either focused only on the evaluation of
(expert-tuned) model-based approaches {[}e.g. 12,13,14{]}, or
exclusively on human forecasts {[}19,20,24,25{]}. In contrast, we
directly compared human and model-based forecasts. This is similar to
the approach taken by {[}11{]}, but extends it in several ways. While
Farrow et al.~only asked for point predictions and constructed a
predictive distribution from these, we asked participants to provide a
full predictive distribution, allowing us to compare human forecasts and
models without any further assumptions, as well as to analyse how humans
quantified their uncertainty. In addition, we compared crowd forecasts
to two semi-mechanistic models informed by basic epidemiological
knowledge of COVID-19, allowing us to assess not only relative
performance but also to analyse qualitative differences between human
judgement and model-based insight. In terms of interpretability of the
results, exact knowledge of our two models, as well as focus on a
limited set of targets and locations was a major advantage of our study
compared to larger studies conducted by the Forecast Hubs
{[}12--15,17{]}.

The strong performance of crowd forecasts in our study is in line with
results from Farrow et al.~who also report strong performance of human
predictions in past Flu challenges despite difficulties to recruit a
large number of participants. The advantage of crowd forecasts we
observed over our semi-mechanistic models is likely in part explained by
the fact that we compared an ensemble of crowd forecasts with single
models. However, this probably explains only part of the difference, and
performance relative to the Hub ensemble strongly suggests that human
insight is valuable when forecasting highly volatile and potentially
hard-to-predict quantities such as case numbers. One potential
explanation is that humans can have access to data that is not available
to or hard to integrate into model-based forecasts. Relatively good
performance of our semi-mechanistic models short-term, but not
longer-term, suggests that model-based forecasts are helpful to
extrapolate from current conditions, but require some form of human
intervention or additional assumptions to inform forecasts when
conditions change over time. This human intervention may be particularly
important when dealing with artefacts in reporting and data anomalies
(and especially when using daily, rather than weekly data). The large
variance in predictions in January in Germany for example (severe
under-prediction followed by severe over-prediction, see Fig
\ref{fig:forecasts-and-truth}A), may in part be caused by the fact that
the renewal model operated on daily data and therefore was susceptible
to fluctuations in daily reporting which have less of an influence on
weekly reporting.

Our results suggest that human intervention may be less beneficial when
forecasting deaths (especially at shorter horizons, when deaths are
largely dependent on already observed cases), which benefits from the
ability to model the delays and exact epidemiological relationships
between different leading and lagged indicators. Relatively good
performance of the convolution model, especially compared to the poor
performance of the renewal model on deaths (which used only deaths to
estimate and predict the effective reproduction number) underlines the
importance of including leading indicators such as cases as a predictor
for deaths.

Given the low number of participants in our study, it is difficult to
generalise conclusions about crowd predictions to other settings. Using
R shiny as a platform for the web application arguably created some
limits to user experience and performance, influencing the number of
participants and potentially creating a self-selection effect.
Motivating forecasters to contribute regularly proved challenging,
especially given that the majority of our participants were from the UK
and may not have been familiar with all relevant details of the
situation in Germany and Poland. On the other hand, R shiny facilitated
quick development and allowed us to provide our crowd forecasting
tooling as an open source R package, meaning that it is available for
others to use, for example in settings like early-stage outbreaks where
model-based forecasts are not available. In light of the relatively
small number of Hub ensemble models, performance of the Hub ensemble is
also difficult to generalise. More research is needed to replicate these
findings and investigate how crowd forecasts compare against the types
of models and model ensembles policy makers use to inform their
decisions.

Our work suggests that crowd forecasts and model-based forecasts could
have different strengths and may be able to complement each other. When
choosing a suitable approach for a given task it is important to take
into account how the output will be used. In this work we focused on
forecasts (which aim to predict future data points whilst accounting for
all factors that might influence them), whereas policy makers might be
more interested in projections (which show what would happen in the
absence of any events that could change the trend) or scenario
modelling. Forecasts may not be a suitable basis for informing policy
decisions, if forecasters already have factored in the expectation of a
future intervention. Model-based approaches can be either forecasts or
projections depending on the assumptions, whereas eliciting projections
that are not influenced by implicit assumptions about the future from
humans may be harder.

Further work should explore the effects of humans refining their
mathematical models or changing model outputs in more detail.
Model-based forecasts could be used as an input to human judgement, with
researchers adjusting predictions generated by models. Seeing a
model-based forecast could help humans calibrate uncertainty better,
while allowing for manual intervention to adapt spurious trend
predictions. Tools need to be developed to facilitate this process at a
larger scale. Human insight could also be used as an input to models.
Such a `hybrid' forecasting approach could for example ask humans to
predict the trend of the effective reproduction number \(R_t\) or the
doubling rate (i.e.~how the epidemic evolves) into the future and use
this to estimate the exact number of cases, hospitalisations or deaths
this would imply. In light of severe overconfidence, yet good
performance in terms of the absolute error, post-processing of human
forecasts to adjust and widen prediction intervals may be another
promising approach. Crowd forecasting in general could benefit greatly
from the availability of tools suitable to appeal to a greater audience.
Given the good performance we and previous authors observed in spite of
the limited resources available and the small number of participants,
this seems worthwhile to further develop and explore.

\clearpage

\section*{References}
\bibliographystyle{plos2015}
\bibliography{germanpolishpaper.bib, references.bib}

\hypertarget{refs}{}
\begin{CSLReferences}{0}{0}
\leavevmode\vadjust pre{\hypertarget{ref-mcgowanCollaborativeEffortsForecast2019}{}}%
\CSLLeftMargin{1. }%
\CSLRightInline{McGowan CJ, Biggerstaff M, Johansson M, Apfeldorf KM,
Ben-Nun M, Brooks L, et al. Collaborative efforts to forecast seasonal
influenza in the {United States}, 2015--2016. Scientific Reports.
2019;9: 683.
doi:\href{https://doi.org/10.1038/s41598-018-36361-9}{10.1038/s41598-018-36361-9}}

\leavevmode\vadjust pre{\hypertarget{ref-reichCollaborativeMultiyearMultimodel2019}{}}%
\CSLLeftMargin{2. }%
\CSLRightInline{Reich NG, Brooks LC, Fox SJ, Kandula S, McGowan CJ,
Moore E, et al. A collaborative multiyear, multimodel assessment of
seasonal influenza forecasting in the {United States}. PNAS. 2019;116:
3146--3154.
doi:\href{https://doi.org/10.1073/pnas.1812594116}{10.1073/pnas.1812594116}}

\leavevmode\vadjust pre{\hypertarget{ref-shamanForecastingSeasonalOutbreaks2012}{}}%
\CSLLeftMargin{3. }%
\CSLRightInline{Shaman J, Karspeck A. Forecasting seasonal outbreaks of
influenza. PNAS. 2012;109: 20425--20430.
doi:\href{https://doi.org/10.1073/pnas.1208772109}{10.1073/pnas.1208772109}}

\leavevmode\vadjust pre{\hypertarget{ref-biggerstaffResultsCentersDisease2016}{}}%
\CSLLeftMargin{4. }%
\CSLRightInline{Biggerstaff M, Alper D, Dredze M, Fox S, Fung IC-H,
Hickmann KS, et al. Results from the centers for disease control and
prevention's predict the 2013--2014 {Influenza Season Challenge}. BMC
Infectious Diseases. 2016;16: 357.
doi:\href{https://doi.org/10.1186/s12879-016-1669-x}{10.1186/s12879-016-1669-x}}

\leavevmode\vadjust pre{\hypertarget{ref-johanssonOpenChallengeAdvance2019}{}}%
\CSLLeftMargin{5. }%
\CSLRightInline{Johansson MA, Apfeldorf KM, Dobson S, Devita J, Buczak
AL, Baugher B, et al. An open challenge to advance probabilistic
forecasting for dengue epidemics. PNAS. 2019;116: 24268--24274.
doi:\href{https://doi.org/10.1073/pnas.1909865116}{10.1073/pnas.1909865116}}

\leavevmode\vadjust pre{\hypertarget{ref-yamanaSuperensembleForecastsDengue2016}{}}%
\CSLLeftMargin{6. }%
\CSLRightInline{Yamana TK, Kandula S, Shaman J. Superensemble forecasts
of dengue outbreaks. Journal of The Royal Society Interface. 2016;13:
20160410.
doi:\href{https://doi.org/10.1098/rsif.2016.0410}{10.1098/rsif.2016.0410}}

\leavevmode\vadjust pre{\hypertarget{ref-colon-gonzalezProbabilisticSeasonalDengue2021}{}}%
\CSLLeftMargin{7. }%
\CSLRightInline{Colón-González FJ, Bastos LS, Hofmann B, Hopkin A,
Harpham Q, Crocker T, et al. Probabilistic seasonal dengue forecasting
in {Vietnam}: {A} modelling study using superensembles. PLOS Medicine.
2021;18: e1003542.
doi:\href{https://doi.org/10.1371/journal.pmed.1003542}{10.1371/journal.pmed.1003542}}

\leavevmode\vadjust pre{\hypertarget{ref-viboudRAPIDDEbolaForecasting2018}{}}%
\CSLLeftMargin{8. }%
\CSLRightInline{Viboud C, Sun K, Gaffey R, Ajelli M, Fumanelli L, Merler
S, et al. The {RAPIDD} ebola forecasting challenge: {Synthesis} and
lessons learnt. Epidemics. 2018;22: 13--21.
doi:\href{https://doi.org/10.1016/j.epidem.2017.08.002}{10.1016/j.epidem.2017.08.002}}

\leavevmode\vadjust pre{\hypertarget{ref-funkAssessingPerformanceRealtime2019}{}}%
\CSLLeftMargin{9. }%
\CSLRightInline{Funk S, Camacho A, Kucharski AJ, Lowe R, Eggo RM,
Edmunds WJ. Assessing the performance of real-time epidemic forecasts:
{A} case study of {Ebola} in the {Western Area} region of {Sierra
Leone}, 2014-15. PLOS Computational Biology. 2019;15: e1006785.
doi:\href{https://doi.org/10.1371/journal.pcbi.1006785}{10.1371/journal.pcbi.1006785}}

\leavevmode\vadjust pre{\hypertarget{ref-delvalleSummaryResults201420152018}{}}%
\CSLLeftMargin{10. }%
\CSLRightInline{Del Valle SY, McMahon BH, Asher J, Hatchett R, Lega JC,
Brown HE, et al. Summary results of the 2014-2015 {DARPA Chikungunya}
challenge. BMC Infectious Diseases. 2018;18: 245.
doi:\href{https://doi.org/10.1186/s12879-018-3124-7}{10.1186/s12879-018-3124-7}}

\leavevmode\vadjust pre{\hypertarget{ref-farrowHumanJudgmentApproach2017}{}}%
\CSLLeftMargin{11. }%
\CSLRightInline{Farrow DC, Brooks LC, Hyun S, Tibshirani RJ, Burke DS,
Rosenfeld R. A human judgment approach to epidemiological forecasting.
PLOS Computational Biology. 2017;13: e1005248.
doi:\href{https://doi.org/10.1371/journal.pcbi.1005248}{10.1371/journal.pcbi.1005248}}

\leavevmode\vadjust pre{\hypertarget{ref-funkShorttermForecastsInform2020}{}}%
\CSLLeftMargin{12. }%
\CSLRightInline{Funk S, Abbott S, Atkins BD, Baguelin M, Baillie JK,
Birrell P, et al. Short-term forecasts to inform the response to the
{Covid-19} epidemic in the {UK}. medRxiv. 2020; 2020.11.11.20220962.
doi:\href{https://doi.org/10.1101/2020.11.11.20220962}{10.1101/2020.11.11.20220962}}

\leavevmode\vadjust pre{\hypertarget{ref-cramerCOVID19ForecastHub2020}{}}%
\CSLLeftMargin{13. }%
\CSLRightInline{Cramer E, Reich NG, Wang SY, Niemi J, Hannan A, House K,
et al. {COVID-19 Forecast Hub}: 4 {December} 2020 snapshot. {Zenodo};
2020.
doi:\href{https://doi.org/10.5281/zenodo.3963371}{10.5281/zenodo.3963371}}

\leavevmode\vadjust pre{\hypertarget{ref-cramerEvaluationIndividualEnsemble2021}{}}%
\CSLLeftMargin{14. }%
\CSLRightInline{Cramer E, Ray EL, Lopez VK, Bracher J, Brennen A,
Rivadeneira AJC, et al. Evaluation of individual and ensemble
probabilistic forecasts of {COVID-19} mortality in the {US}. medRxiv.
2021; 2021.02.03.21250974.
doi:\href{https://doi.org/10.1101/2021.02.03.21250974}{10.1101/2021.02.03.21250974}}

\leavevmode\vadjust pre{\hypertarget{ref-bracherShorttermForecastingCOVID192021}{}}%
\CSLLeftMargin{15. }%
\CSLRightInline{Bracher J, Wolffram D, Deuschel J, Görgen K, Ketterer
JL, Ullrich A, et al. Short-term forecasting of {COVID-19} in {Germany}
and {Poland} during the second wave -- a preregistered study. medRxiv.
2021; 2020.12.24.20248826.
doi:\href{https://doi.org/10.1101/2020.12.24.20248826}{10.1101/2020.12.24.20248826}}

\leavevmode\vadjust pre{\hypertarget{ref-bracherNationalSubnationalShortterm2021}{}}%
\CSLLeftMargin{16. }%
\CSLRightInline{Bracher J, Wolffram D, Deuschel J, Görgen K, Ketterer
JL, Ullrich A, et al. National and subnational short-term forecasting of
{COVID-19} in {Germany} and {Poland}, early 2021. 2021;
2021.11.05.21265810.
doi:\href{https://doi.org/10.1101/2021.11.05.21265810}{10.1101/2021.11.05.21265810}}

\leavevmode\vadjust pre{\hypertarget{ref-europeancovid-19forecasthubEuropeanCovid19Forecast2021}{}}%
\CSLLeftMargin{17. }%
\CSLRightInline{European Covid-19 Forecast Hub. European {Covid-19
Forecast Hub}. 2021 {[}cited 30 May 2021{]}. Available:
\url{https://covid19forecasthub.eu/}}

\leavevmode\vadjust pre{\hypertarget{ref-reichAccuracyRealtimeMultimodel2019}{}}%
\CSLLeftMargin{18. }%
\CSLRightInline{Reich NG, McGowan CJ, Yamana TK, Tushar A, Ray EL,
Osthus D, et al. Accuracy of real-time multi-model ensemble forecasts
for seasonal influenza in the {U}.{S}. PLOS Computational Biology.
2019;15: e1007486.
doi:\href{https://doi.org/10.1371/journal.pcbi.1007486}{10.1371/journal.pcbi.1007486}}

\leavevmode\vadjust pre{\hypertarget{ref-tetlockForecastingTournamentsTools2014}{}}%
\CSLLeftMargin{19. }%
\CSLRightInline{Tetlock PE, Mellers BA, Rohrbaugh N, Chen E. Forecasting
{Tournaments}: {Tools} for {Increasing Transparency} and {Improving} the
{Quality} of {Debate}. Curr Dir Psychol Sci. 2014;23: 290--295.
doi:\href{https://doi.org/10.1177/0963721414534257}{10.1177/0963721414534257}}

\leavevmode\vadjust pre{\hypertarget{ref-atanasovDistillingWisdomCrowds2016}{}}%
\CSLLeftMargin{20. }%
\CSLRightInline{Atanasov P, Rescober P, Stone E, Swift SA,
Servan-Schreiber E, Tetlock P, et al. Distilling the {Wisdom} of
{Crowds}: {Prediction Markets} vs. {Prediction Polls}. Management
Science. 2016;63: 691--706.
doi:\href{https://doi.org/10.1287/mnsc.2015.2374}{10.1287/mnsc.2015.2374}}

\leavevmode\vadjust pre{\hypertarget{ref-hoogeveenLaypeopleCanPredict2020}{}}%
\CSLLeftMargin{21. }%
\CSLRightInline{Hoogeveen S, Sarafoglou A, Wagenmakers E-J. Laypeople
{Can Predict Which Social-Science Studies Will Be Replicated
Successfully}. Advances in Methods and Practices in Psychological
Science. 2020;3: 267--285.
doi:\href{https://doi.org/10.1177/2515245920919667}{10.1177/2515245920919667}}

\leavevmode\vadjust pre{\hypertarget{ref-replicationmarketsReplicationMarketsReliable2020}{}}%
\CSLLeftMargin{22. }%
\CSLRightInline{ReplicationMarkets. Replication {Markets} -- {Reliable}
research replicates\ldots you can bet on it. 2020 {[}cited 13 Oct
2021{]}. Available: \url{https://www.replicationmarkets.com/}}

\leavevmode\vadjust pre{\hypertarget{ref-servan-schreiberPredictionMarketsDoes2004}{}}%
\CSLLeftMargin{23. }%
\CSLRightInline{Servan-Schreiber E, Wolfers J, Pennock DM, Galebach B.
Prediction {Markets}: {Does Money Matter}? Electronic Markets. 2004;14:
243--251.
doi:\href{https://doi.org/10.1080/1019678042000245254}{10.1080/1019678042000245254}}

\leavevmode\vadjust pre{\hypertarget{ref-mcandrewExpertJudgmentModel2020}{}}%
\CSLLeftMargin{24. }%
\CSLRightInline{McAndrew TC, Reich NG. An expert judgment model to
predict early stages of the {COVID-19} outbreak in the {United States}.
medRxiv. 2020; 2020.09.21.20196725.
doi:\href{https://doi.org/10.1101/2020.09.21.20196725}{10.1101/2020.09.21.20196725}}

\leavevmode\vadjust pre{\hypertarget{ref-recchiaHowWellDid2021}{}}%
\CSLLeftMargin{25. }%
\CSLRightInline{Recchia G, Freeman ALJ, Spiegelhalter D. How well did
experts and laypeople forecast the size of the {COVID-19} pandemic? PLOS
ONE. 2021;16: e0250935.
doi:\href{https://doi.org/10.1371/journal.pone.0250935}{10.1371/journal.pone.0250935}}

\leavevmode\vadjust pre{\hypertarget{ref-metaculusPreliminaryLookMetaculus2020}{}}%
\CSLLeftMargin{26. }%
\CSLRightInline{Metaculus. A {Preliminary Look} at {Metaculus} and
{Expert Forecasts}. 22 Jun 2020 {[}cited 30 May 2021{]}. Available:
\url{https://www.metaculus.com/news/2020/06/02/LRT/}}

\leavevmode\vadjust pre{\hypertarget{ref-hypermindHypermindSupercollectiveIntelligence2021}{}}%
\CSLLeftMargin{27. }%
\CSLRightInline{Hypermind. Hypermind \textbar{} {Supercollective}
intelligence for decision makers. {Hypermind}; 2021 {[}cited 13 Oct
2021{]}. Available: \url{https://www.hypermind.com/en/}}

\leavevmode\vadjust pre{\hypertarget{ref-csetforetellCSETForetell2021}{}}%
\CSLLeftMargin{28. }%
\CSLRightInline{CSET Foretell. {CSET Foretell}. 2021 {[}cited 13 Oct
2021{]}. Available: \url{https://www.cset-foretell.com/}}

\leavevmode\vadjust pre{\hypertarget{ref-predictitPredictIt2021}{}}%
\CSLLeftMargin{29. }%
\CSLRightInline{PredictIt. {PredictIt}. 2021 {[}cited 13 Oct 2021{]}.
Available: \url{https://www.predictit.org/}}

\leavevmode\vadjust pre{\hypertarget{ref-heldProbabilisticForecastingInfectious2017}{}}%
\CSLLeftMargin{30. }%
\CSLRightInline{Held L, Meyer S, Bracher J. Probabilistic forecasting in
infectious disease epidemiology: The 13th {Armitage} lecture. Statistics
in Medicine. 2017;36: 3443--3460.
doi:\href{https://doi.org/10.1002/sim.7363}{10.1002/sim.7363}}

\leavevmode\vadjust pre{\hypertarget{ref-crowdforecastr}{}}%
\CSLLeftMargin{31. }%
\CSLRightInline{Bosse NI, Abbott S, EpiForecasts, Funk S.
Crowdforecastr: Eliciting crowd forecasts in r shiny. 2020.
doi:\href{https://doi.org/10.5281/zenodo.4618519}{10.5281/zenodo.4618519}}

\leavevmode\vadjust pre{\hypertarget{ref-covidgermanforecasts}{}}%
\CSLLeftMargin{32. }%
\CSLRightInline{Bosse NI, Abbott S, EpiForecasts, Funk S.
Covid.german.forecasts: Forecasting covid-19 related metrics for the
german/poland forecast hub. 2020. }

\leavevmode\vadjust pre{\hypertarget{ref-ecdcDownloadHistoricalData2020}{}}%
\CSLLeftMargin{33. }%
\CSLRightInline{ECDC. Download historical data (to 14 {December} 2020)
on the daily number of new reported {COVID-19} cases and deaths
worldwide. {European Centre for Disease Prevention and Control}; 14 Dec
2020 {[}cited 30 May 2021{]}. Available:
\url{https://www.ecdc.europa.eu/en/publications-data/download-todays-data-geographic-distribution-covid-19-cases-worldwide}}

\leavevmode\vadjust pre{\hypertarget{ref-rkiRKICoronavirusSARSCoV22021}{}}%
\CSLLeftMargin{34. }%
\CSLRightInline{RKI. {RKI} - {Coronavirus SARS-CoV-2} - {Aktueller
Lage-}/{Situationsbericht} des {RKI} zu {COVID-19}. 2021 {[}cited 30 May
2021{]}. Available:
\url{https://www.rki.de/DE/Content/InfAZ/N/Neuartiges_Coronavirus/Situationsberichte/Gesamt.html}}

\leavevmode\vadjust pre{\hypertarget{ref-forsal.plRozbieznosciStatystykachKoronawirusa2020}{}}%
\CSLLeftMargin{35. }%
\CSLRightInline{Forsal.pl. Rozbieżności w statystykach koronawirusa. 22
tys. przypadków będą doliczone do ogólnej liczby wyników. 2020 {[}cited
30 May 2021{]}. Available:
\url{https://forsal.pl/lifestyle/zdrowie/artykuly/8017628,rozbieznosci-w-statystykach-koronawirusa-22-tys-przypadkow-beda-doliczone-do-ogolnej-liczby-wynikow.html}}

\leavevmode\vadjust pre{\hypertarget{ref-aerzteblattSARSCoV2DiagnostikRKIPasst2020}{}}%
\CSLLeftMargin{36. }%
\CSLRightInline{Ärzteblatt DÄG Redaktion Deutsches.
SARS-CoV-2-Diagnostik: RKI passt Testempfehlungen an. {Deutsches
Ärzteblatt}; 3 Nov 2020 {[}cited 30 May 2021{]}. Available:
\url{https://www.aerzteblatt.de/nachrichten/118001/SARS-CoV-2-Diagnostik-RKI-passt-Testempfehlungen-an}}

\leavevmode\vadjust pre{\hypertarget{ref-golem}{}}%
\CSLLeftMargin{37. }%
\CSLRightInline{Fay C, Guyader V, Rochette S, Girard C. Golem: A
framework for robust shiny applications. 2021. Available:
\url{https://github.com/ThinkR-open/golem}}

\leavevmode\vadjust pre{\hypertarget{ref-shiny}{}}%
\CSLLeftMargin{38. }%
\CSLRightInline{Chang W, Cheng J, Allaire J, Sievert C, Schloerke B, Xie
Y, et al. Shiny: Web application framework for r. 2021. Available:
\url{https://CRAN.R-project.org/package=shiny}}

\leavevmode\vadjust pre{\hypertarget{ref-ourworldindataCOVID19DataExplorer2020}{}}%
\CSLLeftMargin{39. }%
\CSLRightInline{Our World in Data. {COVID-19 Data Explorer}. {Our World
in Data}; 2020 {[}cited 30 May 2021{]}. Available:
\url{https://ourworldindata.org/coronavirus-data-explorer}}

\leavevmode\vadjust pre{\hypertarget{ref-epinow2}{}}%
\CSLLeftMargin{40. }%
\CSLRightInline{Abbott S, Hellewell J, Hickson J, Munday J, Gostic K,
Ellis P, et al. EpiNow2: Estimate real-time case counts and time-varying
epidemiological parameters. -. 2020;-: --.
doi:\href{https://doi.org/10.5281/zenodo.3957489}{10.5281/zenodo.3957489}}

\leavevmode\vadjust pre{\hypertarget{ref-fraserEstimatingIndividualHousehold2007}{}}%
\CSLLeftMargin{41. }%
\CSLRightInline{Fraser C. Estimating {Individual} and {Household
Reproduction Numbers} in an {Emerging Epidemic}. PLOS ONE. 2007;2: e758.
doi:\href{https://doi.org/10.1371/journal.pone.0000758}{10.1371/journal.pone.0000758}}

\leavevmode\vadjust pre{\hypertarget{ref-epiforecasts.ioux2fcovidCovid19TemporalVariation2020}{}}%
\CSLLeftMargin{42. }%
\CSLRightInline{epiforecasts.io/covid. Covid-19: {Temporal} variation in
transmission during the {COVID-19} outbreak. {Covid-19}; 2020 {[}cited
30 May 2021{]}. Available: \url{https://epiforecasts.io/covid/}}

\leavevmode\vadjust pre{\hypertarget{ref-sherrattExploringSurveillanceData2021}{}}%
\CSLLeftMargin{43. }%
\CSLRightInline{Sherratt K, Abbott S, Meakin SR, Hellewell J, Munday JD,
Bosse N, et al. Exploring surveillance data biases when estimating the
reproduction number: With insights into subpopulation transmission of
{Covid-19} in {England}. 2021; 2020.10.18.20214585.
doi:\href{https://doi.org/10.1101/2020.10.18.20214585}{10.1101/2020.10.18.20214585}}

\leavevmode\vadjust pre{\hypertarget{ref-abbottEstimatingTimevaryingReproduction2020a}{}}%
\CSLLeftMargin{44. }%
\CSLRightInline{Abbott S, Hellewell J, Thompson RN, Sherratt K, Gibbs
HP, Bosse NI, et al. Estimating the time-varying reproduction number of
{SARS-CoV-2} using national and subnational case counts. 2020 {[}cited
14 Oct 2021{]}.
doi:\href{https://doi.org/10.12688/wellcomeopenres.16006.1}{10.12688/wellcomeopenres.16006.1}}

\leavevmode\vadjust pre{\hypertarget{ref-kraemer2020epidemiological}{}}%
\CSLLeftMargin{45. }%
\CSLRightInline{Xu B, Gutierrez B, Hill S, Scarpino S, Loskill A, Wu J,
et al. Epidemiological data from the nCoV-2019 outbreak: Early
descriptions from publicly available data. 2020. Available:
\url{http://virological.org/t/epidemiological-data-from-the-ncov-2019-outbreak-early-descriptions-from-publicly-available-data/337}}

\leavevmode\vadjust pre{\hypertarget{ref-rstan}{}}%
\CSLLeftMargin{46. }%
\CSLRightInline{Stan Development Team. RStan: The r interface to stan.
2020. Available: \url{http://mc-stan.org/}}

\leavevmode\vadjust pre{\hypertarget{ref-bracherEvaluatingEpidemicForecasts2021}{}}%
\CSLLeftMargin{47. }%
\CSLRightInline{Bracher J, Ray EL, Gneiting T, Reich NG. Evaluating
epidemic forecasts in an interval format. PLoS Comput Biol. 2021;17:
e1008618.
doi:\href{https://doi.org/10.1371/journal.pcbi.1008618}{10.1371/journal.pcbi.1008618}}

\leavevmode\vadjust pre{\hypertarget{ref-gneiting_strictly_2007}{}}%
\CSLLeftMargin{48. }%
\CSLRightInline{Gneiting T, Raftery AE. Strictly proper scoring rules,
prediction, and estimation. Journal of the American Statistical
Association. 2007;102: 359--378.
doi:\href{https://doi.org/10.1198/016214506000001437}{10.1198/016214506000001437}}

\leavevmode\vadjust pre{\hypertarget{ref-scoringutils}{}}%
\CSLLeftMargin{49. }%
\CSLRightInline{Bosse NI, Abbott S, EpiForecasts, Funk S. Scoringutils:
Utilities for scoring and assessing predictions. 2020.
doi:\href{https://doi.org/10.5281/zenodo.4618017}{10.5281/zenodo.4618017}}

\leavevmode\vadjust pre{\hypertarget{ref-deutschewelleCoronavirusGermanyImpose2020}{}}%
\CSLLeftMargin{50. }%
\CSLRightInline{Deutsche Welle. Coronavirus: {Germany} to impose
one-month partial lockdown \textbar{} {DW} \textbar{} 28.10.2020. 2020
{[}cited 29 Jun 2021{]}. Available:
\url{https://www.dw.com/en/coronavirus-germany-to-impose-one-month-partial-lockdown/a-55421241}}

\end{CSLReferences}

\clearpage

\section*{Acknowledgements}

We thank all forecasters who participated in this study for their
contribution.

\clearpage

\beginsupplement

\section*{Supporting information}

\paragraph*{S1 Table.}
\label{tab:scoring-metrics}

\textbf{Overview of the scoring metrics used.}

\paragraph*{S1 Text.}
\label{txt:details-models}

\textbf{Further details on the semi-mechanistic forecasting models.}

\paragraph*{S1 Fig.}
\label{fig:screenshot}

\textbf{Screenshot of the crowdforecasting app used to elicit predictions (made in June 2021).}

\paragraph*{S2}

\textbf{Table.} \label{tab:score-table-2}
\textbf{Scores for one and two week ahead forecasts.} Scores are cut to
three significant digits and rounded). Note that scores for cases (which
include the whole period from October 12th 2020 until March 1st 2021)
and deaths (which include only forecasts from the 21st of December 2020
on) are computed on different subsets. Numbers in brackets show the
metrics relative to the Hub ensemble (i.e.~the median ensemble of all
other models submitted to the German and Polish Forecast Hub, excluding
our contributions). WIS is the mean weighted interval score (lower
values are better), WIS - sd is the standard deviation of all scores
achieved by a model. Dispersion, over-prediction and under-prediction
together sum up to the weighted interval score. Bias (between -1 and 1,
0 is ideal) represents the general average tendency of a model to over-
or underpredict. 50\% and 90\%-coverage are the percentage of observed
values that fell within the 50\% and 90\% prediction intervals of a
model.

\paragraph*{S3 Table.}
\label{tab:score-table-4}

\textbf{Scores for three and four weeks ahead forecasts.} Scores are cut
to three significant digits and rounded). Note that scores for cases
(which include the whole period from October 12th 2020 until March 1st
2021) and deaths (which include only forecasts from the 21st of December
2020 on) are computed on different subsets. Numbers in brackets show the
metrics relative to the Hub ensemble (i.e.~the median ensemble of all
other models submitted to the German and Polish Forecast Hub, excluding
our contributions). WIS is the mean weighted interval score (lower
values are better), WIS - sd is the standard deviation of all scores
achieved by a model. Dispersion, over-prediction and under-prediction
together sum up to the weighted interval score. Bias (between -1 and 1,
0 is ideal) represents the general average tendency of a model to over-
or underpredict. 50\% and 90\%-coverage are the percentage of observed
values that fell within the 50\% and 90\% prediction intervals of a
model.

\paragraph{S2}

\textbf{Fig.} \label{fig:agg-performance-all-Germany}
\textbf{Visualisation of aggregate performance metrics for forecasts one to four weeks into the future in Germany.}
A, B: mean weighted interval score (WIS, lower indicates better
performance) across horizons. WIS is decomposed into its components
dispersion, over-prediction and under-prediction. C: Empirical coverage
of the 50\% prediction intervals (50\% coverage is perfect). D:
Empirical coverage of the 90\% prediction intervals. E: Dispersion (same
as in panel A, B). Higher values mean greater dispersion of the forecast
and imply ceteris paribus a worse score. F: Bias, i.e.~general
(relative) tendency to over- or underpredict. Values are between -1
(complete under-prediction) and 1 (complete over-prediction) and 0
ideally. G: Absolute error of the median forecast (lower is better). H.
Standard deviation of all WIS values for different horizons

\paragraph{S3}

\textbf{Fig.} \label{fig:agg-performance-all-Poland}
\textbf{Visualisation of aggregate performance metrics for forecasts one to four weeks into the future in Poland.}
A, B: mean weighted interval score (WIS, lower indicates better
performance) across horizons. WIS is decomposed into its components
dispersion, over-prediction and under-prediction. C: Empirical coverage
of the 50\% prediction intervals (50\% coverage is perfect). D:
Empirical coverage of the 90\% prediction intervals. E: Dispersion (same
as in panel A, B). Higher values mean greater dispersion of the forecast
and imply ceteris paribus a worse score. F: Bias, i.e.~general
(relative) tendency to over- or underpredict. Values are between -1
(complete under-prediction) and 1 (complete over-prediction) and 0
ideally. G: Absolute error of the median forecast (lower is better). H.
Standard deviation of all WIS values for different horizons

\paragraph{S4}

\textbf{Fig.} \label{fig:performance-locations}
\textbf{Visualisation of aggregate performance metrics across locations.}
A, B: mean weighted interval score (WIS, lower indicates better
performance) across horizons. WIS is decomposed into its components
dispersion, over-prediction and under-prediction. C: Empirical coverage
of the 50\% prediction intervals (50\% coverage is perfect). D:
Empirical coverage of the 90\% prediction intervals. E: Dispersion (same
as in panel A, B). Higher values mean greater dispersion of the forecast
and imply ceteris paribus a worse score. F: Bias, i.e.~general
(relative) tendency to over- or underpredict. Values are between -1
(complete under-prediction) and 1 (complete over-prediction) and 0
ideally. G: Absolute error of the median forecast (lower is better). H.
Standard deviation of WIS values.

\paragraph{S5}

\textbf{Fig.} \label{fig:performance-locations-rel}
\textbf{Visualisation of aggregate performance metrics across locations in relative terms.}
A, B: mean weighted interval score (WIS) across locations (lower values
indicate better performance). C, D: Empirical coverage of the 50\% and
90\% prediction intervals. E: Dispersion. Higher values mean greater
dispersion of the forecast and imply ceteris paribus a worse score. F:
Bias, i.e.~general (relative) tendency to over- orunderpredict. Values
are between -1 (complete under-prediction) and 1 (complete
over-prediction) and 0 ideally. G: Absolute error of the median
forecast. H. Standard deviation of WIS values.

\paragraph{S6}

\textbf{Fig.} \label{fig:daily-truth}
\textbf{Visualisation of daily report data.} The black line represents
weekly data divided by seven. Data were last accessed through the German
and Polish Forecast Hub on August 21 2021.

\paragraph{S7}

\textbf{Fig.} \label{fig:daily-truth-update}
\textbf{Visualisation of the absolute difference between the daily report data at the time and the data now.}
In Germany, there were zero cases and deaths reported on 2020-10-12, and
only later 2467 cases and 6 deaths were added. Data were last accessed
through the German and Polish Forecast Hub on May 10 2022.

\paragraph{S8}

\textbf{Fig.} \label{fig:weekly-truth-update}
\textbf{Visualisation of the relative difference between the weekly report data at the time and the data now.}
Apart from the data that was retrospectively added on 2020-10-12, data
updates did not have a noticeable effect on weekly data (as shown in the
forecasting application). Data were last accessed through the German and
Polish Forecast Hub on May 10 2022.

\paragraph{S9 Fig.}
\label{fig:forecasts-and-truth-1}

\textbf{Visualisation of forecasts and scores for one week ahead forecasts.}
A, C: Visualisation of 50\% prediction intervals of one week ahead
forecasts against the reported values. Forecasts that were not scored
(because there was no complete set of death forecasts available) are
greyed out. B, D: Visualisation of corresponding WIS.

\paragraph{S10 Fig.}
\label{fig:forecasts-and-truth-3}

\textbf{Visualisation of forecasts and scores for three week ahead forecasts.}
A, C: Visualisation of 50\% prediction intervals of three week ahead
forecasts against the reported values. Forecasts that were not scored
(because there was no complete set of death forecasts available) are
greyed out. B, D: Visualisation of corresponding WIS.

\paragraph{S11 Fig.}
\label{fig:forecasts-and-truth-4}

\textbf{Visualisation of forecasts and scores for three week ahead forecasts.}
A, C: Visualisation of 50\% prediction intervals of four week ahead
forecasts against the reported values. Forecasts that were not scored
(because there was no complete set of death forecasts available) are
greyed out. B, D: Visualisation of corresponding WIS.

\paragraph{S12 Fig.}
\label{fig:distribution-scores-1}

\textbf{Distribution of weighted interval scores for one week ahead forecasts}
A: Distribution of weighted interval scores for one week ahead forecasts
of the different models and forecast targets pooled across locations. B:
Distribution of WIS separate by country.

\paragraph{S13 Fig.}
\label{A: Distribution of weighted interval scores for three week ahead forecasts of the different models and forecast targets pooled across locations. B: Distribution of WIS separate by country.}

\textbf{Distribution of weighted interval scores for three week ahead forecasts}
A: Distribution of weighted interval scores for three week ahead
forecasts of the different models and forecast targets pooled across
locations. B: Distribution of WIS separate by country.

\paragraph{S14 Fig.}
\label{A: Distribution of weighted interval scores for four week ahead forecasts of the different models and forecast targets pooled across locations. B: Distribution of WIS separate by country.}

\textbf{Distribution of weighted interval scores for four week ahead forecasts}
A: Distribution of weighted interval scores for four week ahead
forecasts of the different models and forecast targets pooled across
locations. B: Distribution of WIS separate by country.

\paragraph{S15 Fig.}
\label{fig:distribution-scores-ranks-1}

\textbf{Distribution of model ranks (in terms of WIS) for one week ahead forecasts}
A: Distribution of the ranks (determined by the weighted interval score)
for one week ahead forecasts of the different models and forecast
targets, pooled across locations. B: Distribution of ranks separate by
country.

\paragraph{S16 Fig.}
\label{A: Distribution of the ranks (determined by the weighted interval score) for two week ahead forecasts of the different models and forecast targets, pooled across locations. B: Distribution of ranks separate by country.}

\textbf{Distribution of model ranks (in terms of WIS) for two week ahead forecasts}
A: Distribution of the ranks (determined by the weighted interval score)
for two week ahead forecasts of the different models and forecast
targets, pooled across locations. B: Distribution of ranks separate by
country.

\paragraph{S17 Fig.}
\label{A: Distribution of the ranks (determined by the weighted interval score) for three week ahead forecasts of the different models and forecast targets, pooled across locations. B: Distribution of ranks separate by country.}

\textbf{Distribution of model ranks (in terms of WIS) for three week ahead forecasts}
A: Distribution of the ranks (determined by the weighted interval score)
for three week ahead forecasts of the different models and forecast
targets, pooled across locations. B: Distribution of ranks separate by
country.

\paragraph{S18 Fig.}
\label{A: Distribution of the ranks (determined by the weighted interval score) for four week ahead forecasts of the different models and forecast targets, pooled across locations. B: Distribution of ranks separate by country.}

\textbf{Distribution of model ranks (in terms of WIS) for four week ahead forecasts}
A: Distribution of the ranks (determined by the weighted interval score)
for four week ahead forecasts of the different models and forecast
targets, pooled across locations. B: Distribution of ranks separate by
country.

\paragraph{S19 Fig.}
\label{fig:distribution-scores-differences}

\textbf{Difference in WIS between the Crowd forecast and the Hub ensemble for two week ahead forecasts.}
Values below zero mean better performance of the Crowd forecasts.

\paragraph{S20 Fig.}
\label{fig:distribution-scores-differences-renewal}

\textbf{Difference in WIS between the Crowd forecast and the Renewal model for two week ahead forecasts.}
Values below zero mean better performance of the Crowd forecasts.

\paragraph{S21}

\textbf{Fig.} \label{fig:agg-performance-ensemble-mean}
\textbf{Visualisation of aggregate performance metrics across forecast horizons for the different versions of the Hub mean ensemble.}
``Hub-ensemble'' \textit{excludes} all our models, Hub-ensemble-all
\textit{includes} all of our models, ``Hub-ensemble-realised'' is the
actual hub-ensemble observed in reality, which includes the renewal
model and the crowd forecasts, but ont the convolution model. Values
(except for Bias) are computed as differences to the Hub ensemble which
excludes our contributions. For Coverage, this is an absolute
difference, for other metrics this is a percentage difference. A, B:
mean weighted interval score (WIS) across horizons relative to the Hub
ensemble (lower values indicate better performance). C, D: Empirical
coverage of the 50\% and 90\% prediction intervals minus empirical
coverage observed for the Hub ensemble. E: Dispersion relative to the
dispersion of the Hub ensemble. Higher values mean greater dispersion of
the forecast and imply ceteris paribus a worse score. F: Bias,
i.e.~general (relative) tendency to over- orunderpredict. Values are
between -1 (complete under-prediction) and 1 (complete over-prediction)
and 0 ideally. G: Absolute error of the median forecast relative to the
Hub ensemble. H. Standard deviation of all WIS values for different
horizons relative to the Hub ensemble.

\paragraph{S4}

\textbf{Table.} \label{tab:score-table-ensemble-2}
\textbf{Scores for one and two week ahead forecasts for the different versions of the median ensemble.}
Scores are cut to three significant digits and rounded. Note that scores
for cases (which include the whole period from October 12th 2020 until
March 1st 2021) and deaths (which include only forecasts from the 21st
of December 2020 on) are computed on different subsets. Numbers in
brackets show the metrics relative to the Hub ensemble (i.e.~the median
ensemble of all other models submitted to the German and Polish Forecast
Hub, excluding our contributions). WIS is the mean weighted interval
score (lower values are better), WIS - sd is the standard deviation of
all scores achieved by a model. Dispersion, over-prediction and
under-prediction together sum up to the weighted interval score. Bias
(between -1 and 1, 0 is ideal) represents the general average tendency
of a model to over- or underpredict. 50\% and 90\%-coverage are the
percentage of observed values that fell within the 50\% and 90\%
prediction intervals of a model.

\paragraph{S5 Table.}
\label{tab:score-table-ensemble-4}

\textbf{Scores for three and four week ahead forecasts for the different versions of the median ensemble.}
Scores are cut to three significant digits and rounded. Note that scores
for cases (which include the whole period from October 12th 2020 until
March 1st 2021) and deaths (which include only forecasts from the 21st
of December 2020 on) are computed on different subsets. Numbers in
brackets show the metrics relative to the Hub ensemble (i.e.~the median
ensemble of all other models submitted to the German and Polish Forecast
Hub, excluding our contributions). WIS is the mean weighted interval
score (lower values are better), WIS - sd is the standard deviation of
all scores achieved by a model. Dispersion, over-prediction and
under-prediction together sum up to the weighted interval score. Bias
(between -1 and 1, 0 is ideal) represents the general average tendency
of a model to over- or underpredict. 50\% and 90\%-coverage are the
percentage of observed values that fell within the 50\% and 90\%
prediction intervals of a model.

\paragraph{S6 Table.}
\label{tab:score-table-ensemble-mean-2}

\textbf{Scores for one and two week ahead forecasts for the different versions of the mean ensemble}
Scores are cut to three significant digits and rounded. Note that scores
for cases (which include the whole period from October 12th 2020 until
March 1st 2021) and deaths (which include only forecasts from the 21st
of December 2020 on) are computed on different subsets. Numbers in
brackets show the metrics relative to the Hub mean ensemble (i.e.~the
mean ensemble of all other models submitted to the German and Polish
Forecast Hub, excluding our contributions). WIS is the mean weighted
interval score (lower values are better), WIS - sd is the standard
deviation of all scores achieved by a model. Dispersion, over-prediction
and under-prediction together sum up to the weighted interval score.
Bias (between -1 and 1, 0 is ideal) represents the general average
tendency of a model to over- or underpredict. 50\% and 90\%-coverage are
the percentage of observed values that fell within the 50\% and 90\%
prediction intervals of a model.

\paragraph{S7 Table.}
\label{tab:score-table-ensemble-mean-4}

\textbf{Scores for three and four week ahead forecasts for the different versions of the mean ensemble}
Scores are cut to three significant digits and rounded. Note that scores
for cases (which include the whole period from October 12th 2020 until
March 1st 2021) and deaths (which include only forecasts from the 21st
of December 2020 on) are computed on different subsets. Numbers in
brackets show the metrics relative to the Hub mean ensemble (i.e.~the
mean ensemble of all other models submitted to the German and Polish
Forecast Hub, excluding our contributions). WIS is the mean weighted
interval score (lower values are better), WIS - sd is the standard
deviation of all scores achieved by a model. Dispersion, over-prediction
and under-prediction together sum up to the weighted interval score.
Bias (between -1 and 1, 0 is ideal) represents the general average
tendency of a model to over- or underpredict. 50\% and 90\%-coverage are
the percentage of observed values that fell within the 50\% and 90\%
prediction intervals of a model.

\paragraph{S22 Fig.}
\label{fig:agg-performance-all-late}

\textbf{Visualisation of aggregate performance metrics across forecast horizons (period from December 14th 2020 on)}
From December 14th 2020 on, all models were available. In the original
analysis, cases and deaths were scored on different periods, as the
convolution model was only added later. This sensitivity analysis shows
performance of all models restricted to the period from December 14 2020
until March 1st 2021 where all models were available. A, B: mean
weighted interval score (WIS, lower indicates better performance) across
horizons. WIS is decomposed into its components dispersion,
over-prediction and under-prediction. C: Empirical coverage of the 50\%
prediction intervals (50\% coverage is perfect). D: Empirical coverage
of the 90\% prediction intervals. E: Dispersion (same as in panel A, B).
Higher values mean greater dispersion of the forecast and imply ceteris
paribus a worse score. F: Bias, i.e.~general (relative) tendency to
over- or underpredict. Values are between -1 (complete under-prediction)
and 1 (complete over-prediction) and 0 ideally. G: Absolute error of the
median forecast (lower is better). H. Standard deviation of all WIS
values for different horizons

\paragraph{S8}

\textbf{Table.} \label{tab:score-table-late-2}
\textbf{Scores for one and two week ahead forecasts (sensitivity analysis)}
Scores are cut to three significant digits and rounded. In the original
analysis, cases and deaths were scored on different periods, as the
convolution model was only added later. This table shows performance of
all models restricted to the period from December 14 2020 until March
1st 2021 where all models were available. Numbers in brackets show the
metrics relative to the Hub ensemble (i.e.~the median ensemble of all
other models submitted to the German and Polish Forecast Hub, excluding
our contributions). WIS is the mean weighted interval score (lower
values are better), WIS - sd is the standard deviation of all scores
achieved by a model. Dispersion, over-prediction and under-prediction
together sum up to the weighted interval score. Bias (between -1 and 1,
0 is ideal) represents the general average tendency of a model to over-
or underpredict. 50\% and 90\%-coverage are the percentage of observed
values that fell within the 50\% and 90\% prediction intervals of a
model.

\paragraph{S9 Table.}
\label{tab:score-table-late-4}

\textbf{Scores for three and four week ahead forecasts (sensitivity analysis)}
Scores are cut to three significant digits and rounded. In the original
analysis, cases and deaths were scored on different periods, as the
convolution model was only added later. This table shows performance of
all models restricted to the period from December 14 2020 until March
1st 2021 where all models were available. Numbers in brackets show the
metrics relative to the Hub ensemble (i.e.~the median ensemble of all
other models submitted to the German and Polish Forecast Hub, excluding
our contributions). WIS is the mean weighted interval score (lower
values are better), WIS - sd is the standard deviation of all scores
achieved by a model. Dispersion, over-prediction and under-prediction
together sum up to the weighted interval score. Bias (between -1 and 1,
0 is ideal) represents the general average tendency of a model to over-
or underpredict. 50\% and 90\%-coverage are the percentage of observed
values that fell within the 50\% and 90\% prediction intervals of a
model.

\paragraph{S10 Table.}
\label{tab:table-ensemble-versions}

\textbf{Overview of the models and ensembles used.}

\paragraph{S23 Fig.}
\label{fig:num-forecasters}

\textbf{Number of participants who submitted a forecast over time.}

\paragraph{S24 Fig.}
\label{fig:num-ensemble-members}

\textbf{Number of member models in the official Hub ensemble} This
includes our crowd forecasts and the renewal model. Note that the
renewal model was not included in the ensemble on December 28th 2020.

\paragraph{S25 Fig.}
\label{fig:compare-forecasters}

\textbf{Crowd forecasts and baseline shown in the application for a two week horizon.}
Shown are the median, as well as the 50\% and 90\% prediction intervals
(in order of decreasing opacity). For any given point in time, the
baseline shown in red is what forecasters saw when they opened the app
(the baseline shown was constant across all forecast horizons).

\paragraph*{S1 Acknowledgements.}

\textbf{Members of the CMMID COVID-19 working group}

\nolinenumbers



\end{document}
